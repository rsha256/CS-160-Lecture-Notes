\section{Thursday, April 27th}
\subsection{Last Class... but wait there's more}
CS 160 has their Jacobs Spring Design Showcase at Studio 310 on Session C (2-3:30 PM).

Note that this is the same classroom where regular lecture is held but the date is different: Wednesday May 4th.

\subsection{Final Deliverables}
What should you submit on \textbf{May 10th}?
\begin{itemize}
    \item \textbf{Writeup:} Now that you have a working app, get feedback from \textbf{3} users, write up what you learned and would change.
    \item \textbf{Video:} Record a demo video of the final application.
    \item \textbf{Source Code:} Remind us of the link to your repo, make sure you have a \texttt{README.txt}.
\end{itemize}

\subsection{Takeaways}
\subsubsection{Why UI is Important}
Major part of work for ``real'' programs
\begin{itemize}
    \item Approximately 50\%
\end{itemize}
You will work on “real” software
\begin{itemize}
    \item Intended for people other than yourself
\end{itemize}
Bad user interfaces cost
\begin{itemize}
    \item Money (5\%$\uparrow$ satisfaction $\implies$ up to 85\%$\uparrow$ profits)
    \item Lives
\end{itemize}
User interfaces hard to get right
\begin{itemize}
    \item People are unpredictable
\end{itemize}

\hrulefill

So what should you do then?\\
Answer: \textbf{Iterative Design}

Finally, don't use intuition for design -- instead observe the user in context. \\
Also separate the designer's conceptual model from the users' (which is from experience \& usage).

\subsection{Future Coursework for Undergraduates}
BCDI: Berkeley Certificate in Design Innovation
\begin{itemize}
    \item Interdisciplinary certificate (like minor)
    \item 4 courses, CS160 counts as one of them -- you are 25\% done!
    \item Look at DES INV courses (they all count)
    \item \href{https://bcdi.berkeley.edu}{https://bcdi.berkeley.edu}
\end{itemize}

Decals and student Orgs:
\begin{itemize}
    \item Web Design Decal \href{https://wdd.io}{https://wdd.io}
    \item InnoD Decals \href{https://www.innovativedesign.club}{https://www.innovativedesign.club}
    \item Berkeley Innovation: \href{https://www.berkeleyinnovation.org}{https://www.berkeleyinnovation.org}
    \item Extended Reality @ Berkeley: \href{https://xr.berkeley.edu}{https://xr.berkeley.edu}
\end{itemize}

\subsubsection{Graduate School}
Thinking about graduate school? Check out
\begin{itemize}
    \item Berkeley MDes – Master in Design
    \item Berkeley MEng – Visual Computing
    \item Berkeley MIMS
    \item Berkeley PhD in CS; iSchool
\end{itemize}

Other Universities with top HCI programs: CMU, U Dub, Stanford, GTech, MIT,
UMD, etc.

\subsubsection{What if I am a Graduate student?}
Relevant Graduate Courses at Berkeley
\begin{itemize}
    \item CS294-137 Immersive Computing (Yang, Hartmann)\footnote{only ugrads allowed will be those who have taken CS 160 -- and by petition}
    \item CS294-184 Building User-Centered Programming Tools (Chasins)\footnote{only for PhDs}
    \item NWMEDIA 203 Critical Making (Paulos)
    \item CS260B Topics in HCI Research (Paulos, Hartmann)
    \item INFO 214: User Experience Research (Fadden)
    \item INFO 247: Information visualization and presentation (Hearst)
    \item INFO C262: Tangible User Interfaces (Ryokai)
    \item INFO C265: Interface Aesthetics (Ryokai)
    \item INFO 290: Human-Centered AI (Salehi)
    \item and many more!
\end{itemize}
\begin{important}
INFO 213 is similar to CS160 – only take one
\end{important}

\subsection{Q\&A:}
\begin{shaded}
Q: What is the best way to get into SWE as an M.Eng.

A: Take CS 169
\end{shaded}

\subsection{Course Evaluations}
Please fill out \href{https://course-evaluations.berkeley.edu/}{https://course-evaluations.berkeley.edu/}
