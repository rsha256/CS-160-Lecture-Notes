\section{Thursday, January 19th}
\subsection{The Design Cycle}

\subsubsection{UI Critique: Font Selection}
As a designer, one can always benefit from considering multiple interfaces and their respective benefits/tradeoffs.

When choosing what font you want text to be displayed in on Power point, you ahve 2 options:

\begin{enumerate}
    \item Have a flattened complete list of all combinations of font name, font family, and font weight, rendered in the system font.
    \item Have a dropdown that appears on hover with the family and weight, rendered in the font right there for you to see.
    \begin{itemize}
        \item This gives a visual inconsistency
        \item Most will not be used, so slows down the application, with minimal usage for non-graphic designers
        \item Does not show what numbers/symbols/etc other visual icons look like
    \end{itemize}
\end{enumerate}
This is not a world-ending decision but there are still tradeoffs. Arguably, neither got it right, which to use depends on the usecase and intended users.

\subsection{BCourses Logistics}
Make sure you are keeping up with the assignments listed there :D

\subsection{Where does design fit into the larger process?}
First, realize that there is both stuff before and after Design.

Before: R\&D -- they create the raw material, look into what matches user needs

After: Engineering (sometimes integrated with design, but sometimes separated), Sales -- marketing, feedback loop for re-design

\subsubsection{Oscillations over Project Lifespan}
The Design Cycle has a dual form which cycles over the number of ideas under consideration; however, this originally large number decays in magnitude to 1, as the project timeline progresses.

\subsubsection{Divergent vs Convergent Phases}
Divergent: Start with a lot of different ideas/prototypes

Convergent: Over time, you get more narrow and concrete on the concept of what you want to make.

\subsubsection{Waterfall Model (Software Engineering)}
A linear ``waterfall'' model where you can only go forward (and therefore don't allow anything to go backwards) does not work.

This model was used by the Federal Government (specifically wrt contractors) which is probably a key reason behind why their software sucks.

\begin{important}
This may not seem like that big of a deal, but as you get closer to shipping the project, these mistakes cost exponentially more to fix.
\end{important}

Iterative design de-risks your design cycle.

\subsubsection{Agile Software Development}
In Silicon Valley, you're more malleable to change. This differs from the similarly essenced iterative design in that this is more technical -- this is about writing code.

\subsection{Shopping Cart Video (from abc)}
In Palo Alto, CA, IDEO is a Design committee looked into making a new, better, shopping cart. \href{https://www.youtube.com/watch?v=M66ZU2PCIcM}{https://www.youtube.com/watch?v=M66ZU2PCIcM}

Student thoughts on how well they followed the design cycle in the video:
\begin{itemize}
    \item For the final evaluation video, it seemed kinda rushed.
    \item Evaluation happened earlier with post-it notes and multiple prototypes
\end{itemize}
Here Evaluation happened internal to the team (as opposed to user-tested).

They established themes through montages.

\subsection{Methods}
\subsubsection{Talking to people}
Talking to people is important, whether it be from asking stakeholders or interviewing experts.

\subsubsection{Stakeholder Map}
Being visual is important for designers, realize who is at risk as a consequence of your decision(s).

\subsection{Brainstorming}
\begin{enumerate}
    \item Sharpen the Focus
    \item Playful Rules
    \item Number your Ideas
    \item Build and Jump
    \item The Space Remembers
    \item Stretch Your Mental Muscles
    \item Get Physical
\end{enumerate}

\begin{shaded}
Aim for quantity, but hope for quality.
\end{shaded}
