\section{Thursday, January 26th}
\subsection{Administrative Details}
Midterm Schedule set:
\begin{enumerate}
    \item Tue Feb 28
    \item Tue April 11
\end{enumerate}
\textbf{No Final Exam}

But Final Presentations will be on Wed or Thu during RRR week -- part of Jacobs Institute Design Showcase! (Schedule TBD)

\subsection{ChatGPT vs Google}
Similarity \& Differences between Web Search (Google) and chat-dialogue interaction (ChatGPT):
\begin{itemize}
    \item Search engines give multiple results which you can synthesize an answer from, whereas ChatGPT just gives an answer
    \item Google is helpful if you are in an explanatory phase
    \item Google gives more sources to cite from
    \item A singular answer from ChatGPT can be biased, whereas multiple sources from Google can allow us to diversify
    \item ChatGPT can be more detailed/step-by-step
    \item ChatGPT can answer questions that it hasn't seen before  in its corpus
    \item ChatGPT has memory, whereas search engines don't explicitly use previously queries
    \item Google has Images and other integrations, whereas ChatGPT is text-in text-out
    \item ChatGPT can be confidently incorrect
\end{itemize}

\begin{shaded}
`Go beyond intuition, observe target users in context to inform your design'
\end{shaded}

\subsection{XEROX PARC}
The first ``desktop'' computer came from here (IBM and Macintosh were influenced by it), though they were just a printer/copier company.

\subsubsection{XEROX 8200}
This was sold as a ``just click the button and you will be good to go'' but in reality, users found that it was too complicated.

Ever very smart people (ACM, Turing Award Winners, Chief Scientists at Powerset/Bing) were unable to operate the machine.

\subsubsection{Observation Techniques}
User Research:
\begin{itemize}
    \item Task Analysis
    \item Contextual Inquiry
    \item etc (Ethnography/Cultural Probes/Diary Studies)
\end{itemize}
Goal: Understand User's  Activities in Context

\subsection{Task Analysis}
\subsubsection{Case Study: BART Ticket Machine}
\begin{itemize}
    \item Lots of stickers, text, and numbers which overwhelms the user
    \item People read left-to-right so having earlier tasks more on the left in a noticeable position would be helpful
\end{itemize}

Solution:
\begin{itemize}
    \item Stratify into groups (i.e. tourists vs commuters)
    \item Age varies $\implies$ you cannot make the stand height too high or too low
    \item Make sure it is accessible to people in wheelchairs
\end{itemize}
\begin{important}
But this is wrong!
\end{important}
We should actually be talking to real users (found in BART stations) and ask them directly. You are a student so they will be more likely to talk to you than someone who is trying to sell you something; however, you should still make sure to properly compensate them for their time even if it is only a \$5 Starbucks Giftcard.

\subsubsection{Task Analysis Questions}
\begin{enumerate}
    \item Who is using the system?
    \item What tasks do they now perform?
    \item What tasks are desired?
    \item How are the tasks learned?
    \item Where are the tasks performed?
    \item What's the relationship between user \& data? 
    \item What other tools does the user have?
    \item How do users communicate with each other? 
    \item How often are the tasks performed? 
    \item What are the time constraints on the tasks? 
    \item What happens when things go wrong?
    \item etc
\end{enumerate}

\subsection{Old and New Tasks}
Now we have clipper cards, Paying with a phone, etc.

\subsection{Learning Tasks}
\begin{enumerate}
    \item What do they need to know?
    \item Do they need training?
    \item Experience, level of education and literacy
\end{enumerate}

\subsection{Where is the task}
Are their effects of other people (i.e. privacy concerns)
\subsubsection{Geography of the BART Station}
\begin{itemize}
    \item Loud
    \item Privacy
    \item Lighting is Dim
    \item Musicians and Rituals
\end{itemize}

\subsection{Other Tools}
Smartphones/laptops/Maps are artifacts whose integrations can be relevant for how users communicate.

\subsection{When things go wrong}
Make sure you have backup strategies

\subsubsection{Japanese QR Code Vending Machine}
This is a disaster as it required you to type out what product you want on clicking numbers multiple time + recieve an email 2FA (requires good internet):
\begin{important}
If your machine takes too long to work, then people will not use it. Sometimes people simply cannot use it.
\end{important}

Solution: Make sure tasks are specific.
