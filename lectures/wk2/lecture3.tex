\section{Tuesday, January 24th: Sketching}
\subsection{Logistics: Discussion Signup}
You can view your discussion assignment here: \href{https://tinyurl.com/cs160disc-sp23}{https://tinyurl.com/cs160disc-sp23}

\subsection{Sketching, Brainstorming, Critique}
\subsubsection{Case Study: Tesla Design}
Tesla Cards do stuff differently:
\begin{itemize}
    \item They have a large screen in the middle (in portrait mode -- vertically long)
    \item No longer physical buttons -- all touch screen
\end{itemize}

\subsection{Visions for the future of Car Designs}
\subsubsection{Apple Touchscreen Car Integration (from WWDC)}
All pixels all the time.

Benefits:
\begin{itemize}
    \item Shows more information
    \item Software will get updated
    \item You don't need to read a complex manual to use the car's basic features
    \item Cheaper than manufacturing buttons
\end{itemize}

\subsubsection{INEOS Grenadier}
All physical buttons.

Benefits:
\begin{itemize}
    \item More granularity in changing a knob by only a few degrees
    \item Does not require you to look at it to adjust
    \item Better in Mountain-like areas without WiFi
    \item Can be used with gloves/wet hands whereas a touchscreen cannot
\end{itemize}

\subsection{Sketching}
\subsubsection{Design Journals}
Can be a mixture of many different drawings/UI representations/etc.

See attached Drawing Pad for reference.

\subsubsection{Storyboard}
Disney uses this a lot:
\begin{itemize}
    \item Combines image frames with text to give a story without committing to all details
    \item Saves developer time
    \item Communicates a lot of information/content without being ``good'' drawings
\end{itemize}

\subsection{Brainstorming}
A particular technique for generating a lot of ideas (this is the divergent phase, not the convergent phase).

\begin{enumerate}
    \item Sharpen the Focus
    \begin{itemize}
        \item Make sure you have the right focus -- not too narrow or fuzzy.
        \item Widening your focus can allow you to consider innovative
    \end{itemize}
    
    \item Playful Rules
    \begin{itemize}
        \item Don't prematurely reject ideas because they sound too immature or playful
    \end{itemize}
    
    \item Number your Ideas
    \begin{itemize}
        \item This makes sure that no one person is attached to an idea
    \end{itemize}
    
    \item Build and Jump
    \begin{itemize}
        \item Exploration vs Exploitation
    \end{itemize}
    
    \item The Space Remembers
    \begin{itemize}
        \item Use a \textbf{lot} of space
        \item Giant Post-It Notes
        \item External Spatial Memory for your team (i.e. ``War Rooms'')
    \end{itemize}
    
    \item Stretch Your Mental Muscles
    \begin{itemize}
        \item Do puzzles
        \item Get immersed in the domain (go somewhere irl)
    \end{itemize}
    
    \item Get Physical
    \begin{itemize}
        \item Sketch
        \item Make Models
    \item Act out
    \end{itemize}
\end{enumerate}

\subsection{Critique}
\begin{important}
This is NOT for you to show off how great your project is -- you do not learn anything if you are just told that you did a great job.
\end{important}
It is also important not to insult (give feedback on the \textit{design}, not the \textit{designer} dispassionately), ask for specific alternates (instead of suggesting).

This differs from `Brainstorming' as `Critique' is more of an evaluation exercise, and less of a divergent mechanism.
