\section{Thursday, March 9th}
\begin{shaded}
Empirical studies are very expensive in terms of time and money
\end{shaded}
\subsection{Usability Testing}
    \subsubsection{Inspection Techniques}
    \begin{enumerate}
        \item Cognitive walkthroughs: put yourself in the shoes of the user. Check to see if the path users want to go is clear and logical. 
        
        \item Heuristic Evaluation: assess based on predetermined criteria / heuristics

        \item Other non-inspection techniques (mechanical turk)
    \end{enumerate}

    \subsubsection{Cognitive Walkthrough}
    \begin{enumerate}
        \item Get concrete goal
        \item get actions needed to complete said goal
        \item At each step as
        \begin{enumerate}
            \item Will the users know what to do?
            \item Will the user notice that the correct action is available?
            \item Will the user interpret the application feedback correctly?
        \end{enumerate}
    \end{enumerate}

    \subsection{Example: Find a book in a library}

    \begin{enumerate}
        \item find the library website
        \item find location of said book
        \item complete search form
        \item parse through to find latest edition
        \item click to get to the book's page
        \item find library and call number
    \end{enumerate}

    \subsubsection{Heuristic Evaluation}
    \begin{enumerate}
        \item Heuristic: rules of thumb that describe features of usable systems
        \item Example: Minimize user's memory load
        \begin{enumerate}
            \item get small number (3-5) people together to look at the UI separately
            \item give them list of heuristics
            \item get them to evaluate your design 
            \item aggregate findings of the independent evaluators and discuss
        \end{enumerate}
        \begin{enumerate}
            \item H2-1:Visibility of system status 
            \item H2-2: Match system and real world
            \item H2-3: User control and freedom
            \item H2-4: Consistency and standards
            \item H2-5: Error prevention
            \item H2-6: Recognition rather than recall
            \item H2-7: Flexibility and efficiency of use
            \item H2-8: Aesthetic and minimalist design
            \item H2-9: Help users recognize, diagnose, recover from errors
            \item H2-10: Help and documentation
        \end{enumerate}
    \end{enumerate}

\paragraph{Visibility of system status}
\begin{enumerate}
    \item Spiny ball
    \item buffering circle
    \item provide redundant information for each decision a user makes (do you want to save changes $\implies$ your changes will not be saved if you don't)
\end{enumerate}

\paragraph{Match System and World}
\begin{enumerate}    
\item follow real world conventions
\item pay attention to metaphors
\item Ex: used to drag floppy disk to trash to eject it. Did not line up with what trash usually does (data deletion)
\end{enumerate}

\paragraph{User control and freedom}
\begin{enumerate}
    \item users don't like being trapped
    \item give them freedom to choose to cancel, escape, universal undo (ctrl-z), postpone, etc.
 
\end{enumerate}

\paragraph{Consistency and standards}
\begin{enumerate}
    \item don't violate standards since they will behave according to how they expect things to behave
    \item example: put a word in a rounded rectangle implies its a button. People will click it.
\end{enumerate}

\paragraph{Error prevention}
\begin{enumerate}
    \item Check for errors (are you sure you want to do that)
    \item visual metaphors can help clear up incongruities
    \item types of errors: slips (right plan, bad execution) and mistakes (wrong plan, good execution). 
\end{enumerate}

\paragraph{Recognition over Recall}
\begin{enumerate}
    \item People are good at recognition and bad at recall
    \item Example: open recent files tab 
\end{enumerate}

\paragraph{Flexibility and efficiency of use}
\begin{enumerate}
    \item Experts want to save time
    \item Example: keyboard shortcuts, autocomplete 
\end{enumerate}

\paragraph{Aesthetic and Minimalist Design}
\begin{enumerate}
    \item Don't overload screen with information. Give them "details on demand"
    \item Occam's razor: remove or hide irrelevant or rarely needed info
    \item Present information in natural reading order (read left to right, top to bottom
\end{enumerate}

\paragraph{Help users diagnose recognise }
\begin{enumerate}
    \item When errors happen, help the users answer the "what do I do next?" question
    \item write good error messages that are descriptive and offer solutions
    \item Let them undo at every step  
\end{enumerate}

\paragraph{Provide help and documentation}
\begin{enumerate}
    \item easy to search
    \item focused on tasks
    \item list concrete steps to carry out
    \item not too long
    \item examples \begin{enumerate}
        \item tutorials
        \item reference manuals
        \item tool-tips
        \item "whats this" cursor
        \item search help bar
    \end{enumerate}
\end{enumerate}

\subsection{Heuristic Evaluation Steps}
\begin{enumerate}
    \item Pre-evaluation training
Provide the evaluator with domain knowledge if needed
2) Evaluation
Individuals evaluate interface then aggregate results
Compare interface elements with heuristics
Work in 2 passes
First pass: get a feel for flow and scope
Second pass: focus on specific elements
Each evaluator produces list of problems
Explain why with reference to heuristic or other information
Be specific and list each problem separately

3) Severity rating
Establishes a ranking between problems
Cosmetic, minor, major and catastrophic
First rate individually, then as a group
4) Debriefing
Discuss outcome with design team
Suggest potential solutions
Assess how hard things are to fix
\end{enumerate}

\subsection{Pros and Cons of HE vs User Testing}
\begin{enumerate}
    \item Pros: Much faster, doesn't require interpreting user actions
    \item Cons: HE is less accurate, may find false positives (things that conceptually could be problems but are not actually), and need multiple evaluations to be done for verification (5 is good)
\end{enumerate}
