\section{Thursday, March 16th}
\subsection{Empirical Evaluation}
GPT4 was recently announced and it can be useful for rapid prototyping. You can say what you want in words and almost automatically view a design render.

\subsubsection{Qualitative Empirical Evaluation}
\textbf{Contextual Inquiry}: try to understand user’s tasks and conceptual model\\
\textbf{Usability Studies}: look for critical incidents in interface

These Qualitative Empirical Evaluation methods allow us to:
\begin{itemize}
    \item Understand what is going on
    \item Look for problems
    \item Roughly evaluate usability of interface
\end{itemize}

\myparagraph{Trends from Qualitative data}
\textbf{Grounded Theory} (Glaser, Strauss) is a way to systematically produce
insights (hypotheses or theories) from qualitative study data.

Process:
\begin{enumerate}
    \item  Review the collected data, look for concepts or ideas that emerge repeatedly.
    \item Coding: Tag these concepts using codes (shorthand descriptions). 
    Codes can be revised throughout the process and can also be hierarchically grouped.
    \item Memoing: “The theorizing write-up of ideas about the codes” – relate codes to each other, reflect on the implications
\end{enumerate}

Software tools like MaxQDA can help you here!

\subsubsection{Quantitative Empirical Evaluation}
Use to reliably measure some aspect of interface\\
Compare two or more designs on a measurable aspect\\
Contribute to theory of Human-Computer Interaction

\myparagraph{Approaches}
Collect and analyze user events that occur in natural use\\
Controlled experiments

\myparagraph{Examples of measures}
Time to complete a task, Average number of errors on a task, Users’ ratings of an interface$\dagger$

$\dagger$ You could argue that users’ perception of speed, error rates etc is as
important than their actual values

\subsubsection{Qualitative vs Quantitative}
Qualitative: Faster, less expensive $\to$ esp. useful in early stages of design cycle

Quantitative: Reliable, repeatable result $\to$ scientific method\\
Best studies produce generalizable results

\subsection{Designing Controlled Experiments}
\begin{enumerate}
    \item State a lucid, testable hypothesis
    \item Identify variables\\
    (independent, dependent, control, random)
    \item Design the experimental protocol
    \item Choose user population
    \item Apply for human subjects protocol review
    \item Run pilot studies
    \item Run the experiment
    \item Perform statistical analysis
    \item Draw conclusions
\end{enumerate}

\subsection{Managing Study Participants}
\subsubsection{The Three Belmont Principles}
\textbf{Respect for Persons}\\
Have a meaningful consent process: give information, and let prospective subjects freely chose to participate

\textbf{Beneficience}\\
Minimize the risk of harm to subjects, maximize potential benefits

\textbf{Justice}\\
Use fair procedures to select subjects (balance burdens \& benefits)

\subsection{Ethics}
Always ask the question: could the information from a study have been obtained through some other means?

\subsection{Privacy and Confidentiality}
\textbf{Privacy}: having control over the extent, timing, and circumstances of sharing oneself with others.\\
\textit{What are you asking them to reveal about themselves?}

\textbf{Confidentiality}: the treatment of information that an individual has disclosed with the expectation that it will not be divulged.\\
\textit{How are you keeping the information after it has been shared?}
