\section{Tuesday, October 25th}
We started with the \textit{DTFS and DFT} which works for \textbf{periodic DT} signals.

Then we moved onto \textit{CTFS} which works for \textbf{periodic CT} signals.

Today we will look at the \textit{DTFT} which works for \textbf{aperiodic DT} signals.

Later we will look at the \textit{CTFT} which works for \textbf{aperiodic CT} signals.

\hrulefill

However we have some good news in store!

$h(n)$, which we define as either $h:\mathbb Z\to\mathbb R$ or $h:\mathbb Z\to\mathbb C$, can be written as $H(\omega)$ via \[H(\omega)=\sum_{n=-\infty}^\infty h(n)e^{-i\omega n}\]
which is exactly the DTFT of $h$.

Note that frequency responses will be $2\pi$-periodic instead of $p$-periodic and likewise $x\leftrightarrow H$ and $t\leftrightarrow \omega$.

\begin{align*}
    x(n) &= \frac1{2\pi}\int_{\langle2\pi\rangle} X(\omega) e^{i\omega n} \mathrm d \omega
    &&\text{[Synthesis Eqn.]}
    \\
    X(\omega) &= \sum_{n=-\infty}^\infty x(n) e^{-i\omega n}
    &&\text{[Analysis Eqn.]}
\end{align*}

\subsection{Method 2}
\begin{align*}
    x(t)=
    &\frac1{2\pi} \int_{\langle2\pi\rangle} X(\omega) e^{i\omega n} \mathrm d \omega
    &&\text{Note: can't have dirac delta on boundary}
    \\
    &\quad
    \langle2\pi\rangle
    =[\lambda,\lambda+2\pi]\quad\exists\lambda\in\mathbb R
    &&\text{uncountable set}
    \\
    x(n)
    &=\int_{\langle2\pi\rangle} \underbrace{\frac{\mathrm d \omega}{2\pi} X(\omega)}_{X_k \in \text{CTFT}} e^{-i\omega n}
    &&\text{Spectrum}
\end{align*}

\subsubsection{Method 2: Alternate Explanation}
Starting from:
\begin{align*}
    H(\omega)
    &=\sum_{n=-\infty}^\infty h(n) \underbrace{e^{-i\omega n}}_{\phi_n(\omega)}
    \\
    P_n(\omega) = e^{-i\omega n}
    \\
    H(\omega+2\pi)=H(\omega)\quad\forall\omega\in\mathbb R
    \\
    \phi_n(\omega+2\pi)=e^{-i(\omega+2\pi)n} = e^{-i\omega n} \cancelto1{e^{-i2\pi n}}
    \\
    H = \sum_n h(n) \phi_n
    \\
    \langle H,\phi_\ell\rangle
    =
    \langle \sum_n h(n)\phi_n,\phi_\ell\rangle
    &= 
    \sum_n h(n)\langle \phi_n,\phi_\ell\rangle
\end{align*}

\subsection{Inner Product Definition}
\begin{align*}
    \langle F, G\rangle 
    &\triangleq \int_{\langle2\pi\rangle} F(\omega)G^\ast(\omega)\mathrm d\omega
\end{align*}

\subsubsection{Method 2: Examples}
\begin{align*}
    \langle \phi_n, \phi_\ell \rangle 
    &\triangleq 
    \int_{\lambda}^{\lambda+2\pi} e^{-i\omega n} e^{i\omega\ell} \mathrm d\omega
    \\
    &=
    \int_{\lambda}^{\lambda+2\pi} e^{i\omega(\ell-n)} \mathrm d\omega
    \\
    &= 0 &&\text{after integration}
\end{align*}

Therefore
\begin{align*}
    \phi_n(\omega)
    &=e^{-i\omega n}
    \\
    \langle \phi_n, \phi_\ell \rangle 
    &= 2\pi\delta(n-\ell)
\end{align*}

\hrulefill

\begin{align*}
    % h(n) 
    % &= \frac1{2\pi} \int_{\langle2\pi\rangle}
    % \\
    \langle \phi_n, \phi_\ell \rangle  = 2\pi\delta(n-\ell)
\end{align*}

\subsection{Ideal lowpass filter}
Let us have a freq. response that is 2$\pi$-periodic which is centered at 0 which length $2B$.

\begin{align*}
    h(n)
    &=
    \frac1{2\pi} \int_{-\pi}^\pi H(\omega) e^{i\omega n} \mathrm d\omega
    \\
    &=
    \frac1{2\pi} \int_{-B}^B H(\omega) e^{i\omega n} \mathrm d\omega
    \\
    &=
    \frac1{2\pi} \int_{-B}^B A e^{i\omega n} \mathrm d\omega
    = \frac A{2\pi} \int_{-B}^B e^{i\omega n} \mathrm d\omega
    \\
    &=
    \frac A{2\pi}
    \left[-\frac{(i e^{i n \omega})}n\right]
    _{-B}^B
    \\&=
    \frac A{2\pi}
    \left(-\frac{(i e^{i n B})}n + \frac{(i e^{-i n B})}n\right)
    \\&=
    \frac A{n\pi}
    \left(\frac{e^{i n B} - e^{-i n B}}{2i}\right)
    \\
    h(n)
    &=
    \frac A{2\pi} \sin(Bn)
    &&\text{Note: Not causal}
\end{align*}

Note that $h\not\in\ell^1$ but $h\in\ell^2$ as decaying on the order of $\frac1n$ does not converge but decaying on the order of $\left(\frac1n\right)^2$ does converge so it is square summable.

Note that the DTFT, $X(\omega)$, will still be defined, just discontinuous -- so the analysis equation will not work, we would need to find the DTFT some other way. This is due to $X(\omega)$ being impulsive in nature -- it has dirac delta(s).

\subsection{Different Example}
\begin{align*}
    h(n)
    &\triangleq
    e^{i\omega_0n}\quad 0<\omega_0<\pi
    \\
    X(\omega)
    &= \dots
\end{align*}

Note that $h\not\in\ell^1$ and $h\not\in\ell^2$ as it does not converge.

\hrulefill
